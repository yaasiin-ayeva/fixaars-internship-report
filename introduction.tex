{\fontsize{12pt}{13pt}\selectfont
\section{Description}

JS Morlu est une organisation internationale de premier plan, ayant son siège en Virginie, aux États-Unis. Elle a su se développer sur plusieurs continents, notamment en Afrique, où elle possède des succursales dans des pays tels que le Ghana, la Gambie, et le Kenya. En tant que société globale, JS Morlu emploie une équipe multiculturelle, composée de collaborateurs provenant de divers horizons, incluant des pays d'Asie, d'Afrique, et d'Amérique. 

\vspace{0.5cm}

Cette diversité culturelle enrichit ses capacités à offrir des solutions adaptées aux réalités locales tout en appliquant des normes globales. L'objectif central de JS Morlu est d'aider les entreprises à maximiser leur potentiel et à maintenir une croissance durable dans un environnement de plus en plus compétitif.

\vspace{0.5cm}

Les services spécifiques offerts par JS Morlu comprennent :

\vspace{0.3cm}

\begin{itemize}
    \item \textbf{Services traditionnels (comptabilité, fiscalité, et audits)} :
JS Morlu propose des services essentiels tels que la tenue des comptes, la gestion de la fiscalité et l’audit financier pour aider les entreprises à maintenir une gestion saine de leurs finances.

\vspace{0.3cm}

\item \textbf{Audit interne et assistance à la préparation des audits} :
JS Morlu aide les entreprises à établir des systèmes d’audit interne robustes qui garantissent la transparence et l'efficacité des processus financiers. De plus, elle assiste les organisations dans la préparation à des audits externes pour assurer la conformité aux normes financières.

\vspace{0.3cm}

\item \textbf{Redressement d'entreprise} :
En cas de difficultés financières ou opérationnelles, JS Morlu intervient pour analyser la situation et mettre en place des stratégies de redressement. Ces interventions incluent des réorganisations financières, des ajustements dans la gestion des coûts, et la recherche de nouvelles sources de revenus.

\vspace{0.3cm}

\item \textbf{Distribution de solutions informatiques} :
JS Morlu offre des solutions logicielles adaptées aux besoins des entreprises, notamment dans le domaine de la gestion financière et des ressources humaines, garantissant une intégration fluide des technologies dans les processus d’entreprise.

\vspace{0.3cm}

\item \textbf{Gestion des risques d'entreprise} :
Pour aider les entreprises à anticiper et à gérer les risques, JS Morlu propose des services de conseil en gestion des risques, en identifiant les vulnérabilités et en mettant en œuvre des plans d’atténuation pour protéger les actifs des entreprises.

\end{itemize}

\vspace{0.5cm}

La succursale de JS Morlu qui m'a accueilli est située à Accra, au Ghana, précisément à Lagos Avenue, dans le quartier d'East Legon, à proximité de la renommée galerie d'art \textbf{Nubuke Foundation}. Cette antenne regroupe une équipe d'une douzaine de professionnels, composée de comptables, d'auditeurs et de développeurs, sous la direction de Monsieur \textbf{Bernard Benpong}, Directeur de la succursale.

\chapter{Présentation du projet}

\section{Contexte et Objectifs du Projet}

Le présent cahier des charges détaille les spécifications complètes pour la conception et le développement de l'application "WowSoFast". Cette plateforme Web est conçue pour répondre aux besoins des utilisateurs au Ghana, en offrant une solution moderne et efficace pour la recherche de services locaux. Le but de l'application est de simplifier et de faciliter la connexion entre les particuliers et les professionnels locaux, quel que soit le secteur d'activité ou la taille de l'entreprise.

\vspace{0.5cm}

L'objectif principal de "WowSoFast" est de fournir une solution automatisée qui simplifie le processus de recherche de services, tout en intégrant des fonctionnalités avancées telles que la gestion des profils utilisateurs, la planification de rendez-vous, le système de notation et d'avis, ainsi que d'autres caractéristiques essentielles pour une expérience utilisateur optimale.

\vspace{0.5cm}

Le système est conçu pour être accessible à un large éventail d'entreprises et d'organisations, s'adaptant aux besoins spécifiques de la communauté locale. L'application sera développée en prenant en compte les spécificités culturelles et linguistiques du Ghana, avec une interface utilisateur multilingue prenant en charge les langues locales.

\vspace{0.5cm}

En résumé, "WowSoFast" aspire à devenir la référence en matière de recherche de services locaux au Ghana, en offrant une solution technologiquement avancée, précise et adaptée aux besoins variés des utilisateurs. Le projet vise à simplifier et à rationaliser le processus de recherche de services, contribuant ainsi à une expérience utilisateur exceptionnelle pour les entreprises et les particuliers.
 
\subsection{Flexibilité et Personnalisation}

Une caractéristique essentielle de l'application "WowSoFast" réside dans sa capacité à être flexible et adaptable, afin de répondre de manière précise aux besoins spécifiques de chaque utilisateur. L'application devra permettre une \textbf{configuration personnalisée} de différents types de services, ainsi que l'adaptation à diverses spécificités locales.

\vspace{0.5cm}

La flexibilité de l'application permettra aux utilisateurs de définir et de personnaliser leurs préférences, que ce soit en termes de services recherchés, de disponibilité, de tarification ou d'autres critères pertinents. Elle offrira également la possibilité de s'ajuster aux particularités culturelles et linguistiques du Ghana, en permettant une utilisation facile et intuitive par une population diversifiée.

\vspace{0.5cm}

Cette adaptabilité permettra aux professionnels locaux de personnaliser leurs profils pour refléter leurs compétences uniques, leurs disponibilités et leurs tarifs, tandis que les utilisateurs pourront ajuster leurs critères de recherche selon leurs besoins spécifiques.

\vspace{0.5cm}

Ainsi, "WowSoFast" s'engage à offrir une solution hautement personnalisée et adaptable, permettant aux utilisateurs de configurer l'application selon leurs préférences individuelles, tout en s'alignant sur les caractéristiques spécifiques du marché local au Ghana.

\subsection{Utilisateurs Cibles}

Les utilisateurs cibles de l'application "WowSoFast" sont divers et incluent :

\begin{enumerate}

    \item Individus à la recherche de services : Qu'il s'agisse de n'importe quel service voulu, WowSoFast serait le point de chutte parfait pour trouver le meilleur prestataire au meilleur prix.

    \item Entreprises de Toutes Tailles : Qu'il s'agisse de petites entreprises locales ou de grandes sociétés, "WowSoFast" vise à être une solution adaptable répondant aux besoins variés en matière de services locaux.

    \item Organisations Gouvernementales et Non Gouvernementales : Les entités gouvernementales et les organisations à but non lucratif bénéficieront de la flexibilité et de l'efficacité de l'application pour la gestion des services nécessaires à leurs missions respectives.

    \item Institutions Éducatives : Écoles, collèges, et universités pourront utiliser "WowSoFast" pour la recherche et la gestion de services locaux nécessaires à leurs activités éducatives.

    \item Organisations à But Non Lucratif : Les ONG pourront tirer parti de la plateforme pour simplifier la recherche et la gestion de services nécessaires à leurs projets sociaux et humanitaires.

    \item Autres Entités ou prestataires indépendants : Toute organisation ou entité, qu'elle soit du secteur privé ou public, cherchant à gérer efficacement la recherche et la connexion avec des services locaux peut bénéficier de l'application.
    
\end{enumerate}

Ainsi, "WowSoFast" vise à répondre aux besoins d'un large éventail d'utilisateurs, contribuant ainsi à simplifier et à améliorer la gestion des services locaux au Ghana, quel que soit le domaine d'activité ou la taille de l'organisation.


\subsection{Intérêts}

Le développement de l'application "WowSoFast" est motivé par la volonté des utilisateurs, entreprises et organisations au Ghana, de simplifier et d'optimiser le processus de recherche et de gestion de services locaux. Les objectifs principaux incluent :

\begin{enumerate}
    \item Réduction des Frictions : Minimiser les obstacles dans la recherche de services en offrant une plateforme intuitive et facile à utiliser.

    \item Efficacité Opérationnelle : Accroître l'efficacité opérationnelle en facilitant la connexion rapide entre les particuliers et les professionnels locaux.

    \item Conformité aux Exigences Locales : Garantir la conformité aux spécificités locales du marché ghanéen, en tenant compte des langues, des coutumes, et des règlements propres au pays.

    \item Amélioration de l'Expérience Utilisateur : Faciliter la recherche de services et la gestion des rendez-vous pour les utilisateurs, tout en offrant une expérience utilisateur agréable.

    \item Optimisation des Processus : Rationaliser la gestion des services locaux en offrant des fonctionnalités avancées telles que la gestion des profils utilisateurs, la planification de rendez-vous, et les systèmes de notation et d'avis.

    \item Large Choix de Services \& Facilité de Recherche  : Accéder rapidement à une diversité de services locaux, couvrant une gamme étendue de besoins, allant des services professionnels aux services personnels, qualifié grâce à des fonctionnalités de recherche avancées et des filtres spécifiques.

    \item Transparence des Profils : Consulter des profils détaillés des professionnels, incluant leurs compétences, tarifs, et les avis d'autres utilisateurs pour prendre des décisions éclairées.

    \item Économies de Temps et d'Efforts : Éviter les tracas liés à la recherche manuelle de services en bénéficiant d'une plateforme centralisée et conviviale.

    \item Satisfaction Client : Expérimenter une satisfaction accrue en recevant des services de qualité correspondant à leurs besoins spécifiques.

    \item Confiance et Fiabilité : Sélectionner des professionnels évalués par d'autres utilisateurs, renforçant ainsi la confiance dans la qualité des services.

    \item Notifications Personnalisées : Recevoir des notifications personnalisées sur les offres de services, les promotions, et les disponibilités des professionnels.

    \item Économies Financières : Bénéficier de tarifs compétitifs et transparents, évitant les surprises financières grâce à une information claire sur les coûts.

\end{enumerate}

En répondant à ces besoins, "WowSoFast" vise à simplifier la vie quotidienne des utilisateurs en offrant une solution moderne et adaptée, contribuant ainsi à la croissance de l'économie locale au Ghana. L'application aspire à devenir un outil essentiel pour la connectivité entre les particuliers et les professionnels, créant ainsi une communauté dynamique de services locaux.

\section{Fonctionnalités Principales}

\begin{itemize}
    \item \textbf{Recherche de Services Locaux :} Les utilisateurs pourront effectuer des recherches ciblées pour trouver des professionnels locaux correspondant à leurs besoins spécifiques, que ce soit pour des services professionnels, personnels, éducatifs, ou autres.

    \item \textbf{Gestion des Profils Utilisateurs :} Les professionnels locaux auront la possibilité de créer des profils détaillés mettant en avant leurs compétences, expériences, tarifs, horaires de disponibilité, et recevront des évaluations de la part des utilisateurs.

    \item \textbf{Système de Notation et d'Avis :} Les utilisateurs auront la possibilité de laisser des avis et des notations sur les professionnels, créant ainsi une communauté transparente où la qualité des services est mise en avant.

    \item \textbf{Configuration Personnalisée des Services :} Les utilisateurs pourront configurer leurs préférences en fonction de leurs besoins spécifiques, que ce soit en termes de services recherchés, de critères de sélection, ou de disponibilités.

    \item \textbf{Notifications Personnalisées :} Les utilisateurs recevront des notifications personnalisées sur les offres de services, les promotions, les disponibilités des professionnels, et d'autres informations pertinentes.

\end{itemize}

Ces fonctionnalités garantissent une expérience utilisateur complète et adaptée, simplifiant ainsi la recherche et la gestion de services locaux.

\chapter{Technologies \& Contraintes technique }

\section{Choix Technologiques}
Le choix des technologies pour ce projet revêt une importance cruciale, car il déterminera la base technique sur laquelle le système sera construit. Ces choix techniques seront en parfaite cohérence avec les spécifications techniques de l'entreprise et seront soigneusement sélectionnés pour garantir l'efficacité de l'exécution des tâches attribuées. Les critères de sélection des technologies incluront leur pertinence par rapport aux objectifs du projet, leur compatibilité avec l'infrastructure technologique existante de l'entreprise, la possibilité d'acquérir des compétences transférables pour les stagiaires et la capacité à maintenir efficacement le projet sur le long terme.
\\
Dans cette optique, voici les stacks techniques actuellement envisagés pour les différentes composantes du système : 
\subsection{Backend}
\begin{itemize}
    \item \textbf{Langages de programmation}: Deux options sont à l'étude, à savoir \textbf{TypeScript} utilisant \textit{Node.js} ou \textbf{PHP}. Le choix dépendra de divers facteurs, notamment la familiarité de l'équipe avec ces langages et leur adéquation aux besoins du projet.
    \item \textbf{Framework}: Deux frameworks sont également envisagés, à savoir \textbf{ExpressJS} et \textbf{Laravel}. ExpressJS est couramment utilisé pour des applications Node.js, tandis que Laravel est une option solide pour les projets PHP. Le choix sera basé sur la compatibilité avec les besoins spécifiques du projet.;
    \item \textbf{Moteur de Base de données}: Le choix se portera sur une base de données relationnelle, avec des options telles que \textbf{PostgreSQL} ou MySQL, ou éventuellement une base de données NoSQL telle que \textbf{MongoDB}. Le choix dépendra des exigences de stockage de données du projet.
\end{itemize}

\subsection{Frontend}
\begin{itemize}
    \item \textbf{Bibliothèques \& Langages JavaScript}: Les stagiaires auront le choix entre React avec TypeScript ou JQuery pour la création d'interfaces utilisateur interactives. Le choix dépendra des compétences et des préférences de l'équipe de développement.
    \item \textbf{Styles}: Pour la conception et le style de l'interface utilisateur, deux options sont envisagées : Tailwind CSS et Bootstrap. Le choix sera basé sur la conception visuelle et les exigences de l'interface utilisateur.
    \item \textbf{Gestion d'État}: Si React est sélectionné comme bibliothèque d'interface utilisateur, Redux sera utilisé pour la gestion de l'état de l'application, garantissant ainsi une gestion transparente des données et de l'interface utilisateur.
\end{itemize}

\subsection{APIs et Services Pertinents }

\begin{itemize}
    \item \textbf{Google Maps API :} Intégration de l'API Google Maps pour fournir des fonctionnalités de localisation, de cartographie et de navigation pour les utilisateurs à la recherche de services locaux.
    
    \item \textbf{Auth0 API :} Utilisation de l'API Auth0 pour la gestion sécurisée de l'authentification et de l'autorisation des utilisateurs, assurant ainsi la protection des données sensibles.
    
    \item \textbf{SendGrid API :} Intégration de l'API SendGrid pour la gestion des notifications par e-mail, assurant une communication efficace avec les utilisateurs pour les rappels de rendez-vous, les mises à jour, etc.
\end{itemize}

\vspace{0.5cm}

L'intégration de ces APIs dans le projet WowSoFast permettra d'optimiser la fonctionnalité, la sécurité et l'efficacité du système, offrant ainsi une expérience utilisateur complète et fiable.

\vspace{0.5cm}

En somme, le choix des technologies est une décision stratégique qui sera prise en fonction des besoins spécifiques du projet. Il s'agit de sélectionner des outils technologiques qui soutiendront efficacement la création, le déploiement et la maintenance du système, tout en permettant une évolutivité future en fonction des besoins changeants de l'entreprise.

\section{Contraintes Techniques}
Les contraintes techniques sont des paramètres essentiels qui seront strictement appliqués aux stagiaires dans le but de garantir le succès et la qualité du projet WowSoFast. Ces contraintes sont fondamentales pour garantir la cohérence, la sécurité et la facilité de maintenance de l'application. Elles contribueront à orienter les choix et les décisions techniques tout au long du développement.

\subsection{Sécurité des Données}
La sécurité des données est une préoccupation majeure dans le développement du Système. Les stagiaires devront respecter les bonnes pratiques de sécurité pour protéger les informations sensibles des employés et de l'entreprise. Les contraintes seront les suivantes :

\begin{enumerate}
    \item \textbf{Chiffrement des Données :} Toutes les données sensibles, telles que les informations personnelles et les informations financières, doivent être stockées et transmises de manière sécurisée à l'aide de protocoles de chiffrement appropriés.
    \item \textbf{Gestion des Authentifications :} Les mécanismes d'authentification robustes, tels que l'authentification à deux facteurs (2FA), doivent être mis en place pour garantir le fait que, seuls les utilisateurs autorisés auront accès au système.
    \item \textbf{Protection contre les Attaques :} Les stagiaires devront mettre en œuvre des mesures de sécurité contre les attaques courantes telles que les injections SQL, les attaques XSS (Cross-Site Scripting) et les attaques CSRF (Cross-Site Request Forgery).
\end{enumerate}

\subsection{Scalabilité et Performances}
Le Projet WowSoFast devra être capable de s'adapter à la croissance des montées en charges. Les contraintes incluent :
\begin{enumerate}
    \item \textbf{Scalabilité :} L'architecture de l'application doit être conçue de manière à permettre une évolution facile, en ajoutant des fonctionnalités ou en gérant une augmentation des pics de requêtes
    \item \textbf{Optimisation des Performances :} Les stagiaires devront mettre en œuvre des pratiques d'optimisation des performances pour garantir que le système reste réactif même en cas de traitement de volumes importants de données. À cet effet, un benchmark sera réalisé lors de chaque phase de test pour évaluer les performances de l'application sous diverses charges de travail.
\end{enumerate}

\subsection{Documentation et Maintenance}
La documentation et la maintenance sont essentielles pour assurer la pérennité du projet. Les contraintes comprennent :
\begin{enumerate}
    \item \textbf{Une Documentation Complète :} Les stagiaires devront construire une documentation claire, complète et à jour de l'architecture, du code source et des procédures d'exploitation.
    \item \textbf{Tests et Déploiement :} Des procédures de test rigoureux doivent être faites avant chaque déploiement pour minimiser les erreurs en production.
    \item \textbf{Gestion des Erreurs :} Une gestion efficace des erreurs doit être mise en place pour permettre une détection rapide des problèmes et leur résolution.
\end{enumerate}

\vspace{1cm}

En respectant ces contraintes techniques, les stagiaires contribueront à la création d'un System robuste, sûr et évolutif qui répondra aux besoins de l'entreprise tout en leur permettant d'acquérir une expérience d'apprentissage précieuse.

\chapter{Methode de travail}
\section{Collaboration et Méthodologie Agile}

Pour mener à bien le projet WowSoFast, nous mettons en place une collaboration efficace et une méthodologie agile. Cette approche flexible s'adapte aux besoins changeants du projet. Voici ce que cela implique :

\subsection{Méthodologie Agile}

\begin{itemize}
  \item Travailler en sprints avec des itérations régulières pour livrer les fonctionnalités.
  \item Organiser des réunions Scrum pour suivre la progression, discuter des obstacles et ajuster le plan.
  \item Maintenir un backlog de fonctionnalités avec des priorités définies en fonction des besoins.
  \item Effectuer des démonstrations à la fin de chaque sprint pour intégrer les commentaires des parties prenantes.
  \item Utiliser des itérations courtes pour maintenir un rythme soutenable et réagir rapidement aux changements.
\end{itemize}

\subsection{BitBucket et Jira comme Plateformes de Collaboration}

\begin{itemize}
  \item Gestion du Code Source : Utilisation de BitBucket pour héberger et gérer le code source du projet de manière centralisée, offrant des fonctionnalités robustes pour le contrôle de version.
  
  \item Suivi des Problèmes et des Tâches : Utilisation de Jira pour la gestion complète des problèmes, des tâches et du suivi de la production, permettant une coordination efficace au sein de l'équipe.
  
  \item Collaboration en Temps Réel : Facilitation de la collaboration en temps réel entre les membres de l'équipe via BitBucket, permettant le partage de code, les discussions sur les modifications, et la collaboration en ligne.
  
  \item Révisions par les Pairs : Mise en place de révisions par les pairs directement sur BitBucket, intégrant les fonctionnalités de Jira pour garantir une approche collaborative et une qualité de code constante.
  
  \item Intégration Continue (CI) : Configuration de l'intégration continue avec BitBucket Pipelines, assurant une exécution automatisée des tests à chaque modification du code et facilitant la livraison continue.
  
  \item Gestion des Versions : Utilisation des fonctionnalités de gestion des versions de BitBucket pour travailler de manière sécurisée sur différentes branches, favorisant ainsi le développement parallèle et la fusion sécurisée des fonctionnalités.
\end{itemize}

L'adoption de BitBucket et Jira comme plateformes de collaboration garantit une gestion complète du cycle de vie du développement, du suivi des tâches à la gestion du code source, offrant ainsi un environnement optimal pour la réussite du projet.

\subsection{Slack comme Plateforme de Collaboration pour WowSoFast}

\begin{itemize}
  \item Communication Instantanée : Slack offre une communication en temps réel, favorisant une collaboration fluide et rapide entre les membres de l'équipe WowSoFast.

  \item Organisation par Canaux : La création de canaux dédiés sur Slack permet d'organiser les discussions par thèmes spécifiques tels que le développement, la gestion de projet, le support client, facilitant ainsi une communication structurée.

  \item Coordination des Équipes : Les canaux spécifiques pour chaque équipe (développement, marketing, support, etc.) simplifient la coordination des activités et permettent aux membres de rester informés des avancées dans leur domaine.

  \item Notifications Personnalisées : Les membres peuvent configurer des notifications spécifiques pour les canaux importants, garantissant ainsi qu'aucune information cruciale ne soit manquée.

  \item Partage de Ressources : Facilité de partage de fichiers, documents et liens pertinents directement sur Slack, centralisant les ressources nécessaires à la progression du projet.

  \item Intégration d'Outils Externes : Intégration d'outils tiers, tels que des bots ou des applications spécifiques à WowSoFast, pour automatiser des tâches, récupérer des informations clés, ou simplifier certains processus.

  \item Archivage des Discussions : L'historique des conversations sur Slack permet de consulter et de référencer les discussions passées, assurant ainsi une traçabilité et une compréhension continue des décisions prises.

  \item Réunions Virtuelles : Slack offre la possibilité d'organiser des réunions virtuelles directement depuis la plateforme, facilitant ainsi la coordination et la communication au sein de l'équipe, surtout si les membres travaillent à distance.

  \item Esprit d'Équipe : Les canaux dédiés à la détente et à la convivialité favorisent un esprit d'équipe fort, contribuant à un environnement de travail agréable.

  \item Sécurité des Données : Slack met l'accent sur la sécurité, garantissant la confidentialité des échanges au sein de l'équipe WowSoFast.
\end{itemize}

En incorporant Slack comme plateforme de collaboration, l'équipe WowSoFast bénéficiera d'un moyen efficace et complet pour faciliter la communication, la coordination des équipes, et la gestion des ressources, contribuant ainsi au succès du projet.
\vspace{1cm}

Cette approche combinée de méthodologie agile et d'utilisation de GitHub et Slack assure une collaboration efficace, un suivi précis du projet et garantit la qualité du code tout au long du développement du systeme.

\vspace{1.5cm}

\section*{Conclusion}

En conclusion, le projet WowSoFast se profile comme une initiative innovante destinée à simplifier et à révolutionner la manière dont les utilisateurs recherchent et interragissent avec des services locaux. En intégrant des fonctionnalités avancées telles que la recherche ciblée, la gestion des profils utilisateurs, la planification de rendez-vous, et le système de notation, WowSoFast vise à créer une plateforme complète et conviviale pour répondre aux besoins diversifiés de la communauté.

\vspace{0.5cm}

L'utilisation de GitHub comme plateforme de collaboration offre une base solide pour la gestion efficace du développement, garantissant un suivi précis du code source, des problèmes, et des tâches. L'intégration de Slack complète cet écosystème en facilitant la communication en temps réel, la coordination des équipes, et le partage transparent des ressources.

\vspace{0.5cm}

En alignant le projet sur des valeurs telles que la flexibilité, la personnalisation, et l'adaptabilité locale, WowSoFast vise à devenir la référence en matière de recherche de services locaux au Ghana. En unissant ces éléments, le projet aspire à créer une expérience utilisateur exceptionnelle tout en contribuant à la croissance et au dynamisme de la communauté locale.

\vspace{0.5cm}

L'équipe WowSoFast est impatiente de voir ce projet prendre vie et est convaincue qu'il deviendra un pilier essentiel pour la connectivité entre les particuliers et les professionnels locaux au Ghana et partout ailleurs. Ce cahier de charges sert de fondement solide pour le développement et la mise en œuvre réussie de WowSoFast, et marque le début d'une nouvelle ère dans la gestion des services locaux.

}
